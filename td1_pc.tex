%st&pdflatex
\documentclass[12pt, a4paper,oneside]{article} %article pour voir la todolist
% \documentclass[12pt, a4paper,oneside]{amsart} %amsart pour faire joli
\usepackage{amssymb, amsfonts, mathrsfs}
%\usepackage{a4wide, enumerate,} %test
% \usepackage{showkeys} % for work version
\usepackage[textwidth=1in,backgroundcolor=white,linecolor=white,bordercolor=white,textcolor=red]{todonotes}
%\usepackage{a4wide}

\usepackage{glossaries}
\usepackage[bookmarks]{hyperref} % in that order, to not have a link with every
				% abreviation
\usepackage{tikz}
\usepackage{tikz-cd}	

\newcommand{\g}[1]{\gls{#1}}

\newglossaryentry{graphs} {
	name={\ensuremath{\Theta}},
	description={the compact family of graphs}
}
\newglossaryentry{graphs_dual} {
	name={\ensuremath{\Theta^{*} }},
	description={the compact family of dual graphs}
}

\input{../these/preamble.tex}

\title{Corrections feuille 1 PC}
\author{Florestan {Martin-Baillon}}

\begin{document}

\maketitle

Les deux premiers exercices
sont essentiellement une utilisation
de l'identité remarquable "carré d'une somme":
\begin{equation}
	(a+b)^2
	=
	a^2 + 2ab + b^2,
\end{equation}
qu'il faut savoir utiliser dans les deux sens:
à la fois pour développer un carré,
et pour le factoriser.

\section{Correction exercice 1}

On doit montrer que
\begin{equation}
	(a^2 + b^2)
	(c^2 + d^2)
	=
	(ac + bd)^{2} 
	+
	(ad - bc)^2
	.
\end{equation}
On développe le membre de gauche:
\begin{align}
	(a^2 + b^2)
	(c^2 + d^2)
	=
	a^2 c^2
	+
	a^2 d^2
	+
	b^2 c^2
	+
	b^2 d^2
	,
\end{align}
et on développe le membre de droite
en utilisant deux fois l'identité remarquable:
\begin{align}
	(ac + bd)^{2} 
	+
	(ad - bc)^2
	&=
	a^2 c^2
	+
	b^2 d^2
	+
	2 acbd
	+
	a^2 d^2
	+
	b^2 c^2
	-
	2 adbc
	\\
	&=
	a^2 c^2
	+
	b^2 d^2
	+
	a^2 d^2
	+
	b^2 c^2,
\end{align}
et on constate que ces deux quantités sont égales.

\section{Correction exercice 2}

(1) On doit montrer l'inégalité
\begin{equation}
	x^2 + y^2 \ge 2xy
	.
\end{equation}
On raisonne par équivalence: c'est à dire qu'on part
de l'énoncé que l'on veut montrer
et on exprime des énoncés équivalents à celui-ci
jusqu'à trouver un énoncé que l'on sait être vrai.
\begin{align}
	x^2 + y^2 & \ge 2xy
	\\
	& \eq 
	\\
	x^2 + y^2 - 2xy & \ge 0
	\\
	& \eq \com{Factorisation de l'identité remarquable}
	\\
	(x - y)^2 & \ge 0
	.
\end{align}
La dernière inégalité est vrai,
car le carré d'un nombre réel est toujours positif.
Par équivalence, l'inégalité dont on est partie est vrai.

(2) Même raisonnement.
On suppose $ x > 0 $.
\begin{align}
	x + \frac{1}{x}  & \ge 2
	\\
	& \eq
	\\
	x + \frac{1}{x} - 2  & \ge 0
	\\
	& \eq \com{Mise au même dénominateur}
	\\
	\frac{x^2 + 1 - 2x}{x} & \ge 0
	\\
	& \eq \com{Factorisation de l'identité remarquable}
	\\
	\frac{(x-1)^2}{x} & \ge 0
	.
\end{align}
La dernière inégalité est vrai,
car $ (x-1)^2 $ est un carré, donc positif
et $ x $ est positif.
Donc l'inégalité de départ est vrai par équivalence.

(3) Même raisonnement. 
\begin{align}
	xy  & \le (\frac{x+y}{2})^2
	\\
	& \eq \com{On sort le $\frac{1}{2}$ du carré}
	\\
	xy  & \le \frac{1}{4}(x+y)^2
	\\
	& \eq
	\\
	4 xy  & \le (x+y)^2
	\\
	& \eq
	\\
	0  & \le (x+y)^2  - 4 xy
	\\
	& \eq \com{On développe l'identité remarquable}
	\\
	0  & \le x^2 + y^2 + 2xy  - 4 xy
	\\
	& \eq
	\\
	0  & \le x^2 + y^2 - 2 xy
	\\
	& \eq \com{On factorise l'identité remarquable}
	\\
	0  & \le (x-y)^2
	.
\end{align}
La dernière inégalité est vrai, toujours
parce qu'un carré est positif,
donc la première est vrai.

(4) Cette inégalité est plus astucieuse.
On se rappelle qu'à la question (1) on a montré:
\begin{equation}
	x^2 + y^2 \ge 2xy
	.
\end{equation}
On peut aussi appliquer cette inégalité à
$ x $ et $ z $, ainsi qu'à $ y $ et $ z $
et on obtient:
\begin{align}
	x^2 + y^2 & \ge 2xy
	\\
	x^2 + z^2 &\ge 2xz
	\\
	y^2 + z^2 &\ge 2yz
	.
\end{align}
Maintenant on somme ces trois inégalités,
pour obtenir:
\begin{align}
	x^2 + y^2
	+ x^2 + z^2
	+ y^2 + z^2 
	& \ge 2xy + 2xz + 2 yz
	\\
	& \eq
	\\
	2 ( x^2 + y^2 + z^2)
	& \ge
	2 (xy+xz + yz)
	\\
	& \eq
	\\
	 x^2 + y^2 + z^2
	& \ge
	xy+xz + yz
	.
\end{align}

\section{Exercice 3}

On doit montrer l'énoncé $ A \Rightarrow B $.
On sait que cet énoncé est équivalent à sa \emph{contraposé}:
$ \text{non } (B) \Rightarrow \text{non } (A) $.
On va montrer la contraposé (parce que c'est, dans ce cas, plus facile).
Écrivons les énoncés
$ \text{non } (A) $ 
et 
$ \text{non } (B) $:

Comme $ (A) $ est
\begin{equation}
	\forall \varepsilon > 0, 0 \le x \le \varepsilon
	,
\end{equation}
et comme par hypothèse
$ x \ge 0 $,
$ (A) $ est équivalent à
\begin{equation}
	\forall \varepsilon > 0, x \le \varepsilon
	,
\end{equation}
On en déduit que l'énoncé 
$ \text{non } (A) $ 
est
\begin{equation}
	\exists \varepsilon > 0, x > \varepsilon
	.
\end{equation}
En effet, on se rappelle que la négation
d'une proposition de la forme
"pour tout $ x $, $ x $ vérifie la propriété $ P $ "
est la proposition
"il existe un $ x $ qui ne vérifie pas la propriété $ P $",
c'est à dire la proposition qui exprime
l'existence d'un \emph{contre-exemple}.

L'énoncé $ (B) $ est
\begin{equation}
	x = 0
\end{equation}
donc sa négation est
\begin{equation}
	x \neq 0
	.
\end{equation}
On doit donc montrer l'implication:
\begin{equation}
	x \neq 0
	\Rightarrow
	\exists \varepsilon > 0, x > \varepsilon
	.
\end{equation}
Traduit en langage courant, cette implication dit:
si $ x $ n'est pas nul, il existe un $ \varepsilon $
strictement positif qui est strictement plus petit que
$ x $. Montrons cela:

Soit $ x \neq 0 $. Par hypothèse,
$ x \ge 0 $, donc $ x > 0 $.
On a $ x > \frac{1}{2} x $;
en effet:
\begin{align}
	x > \frac{1}{2} x
	\eq
	1 > \frac{1}{2}
	,
\end{align}
où l'on a simplifié par $ x $,
ce qui est licite car $ x > 0 $,
et la dernière inégalité est évidemment vrai.
Donc si on pose $ \varepsilon = \frac{1}{2} x $,
on a $ \varepsilon > 0 $ et $ \varepsilon \le x $,
ce qui donne bien le contre-exemple recherché.

\section{Exercice 4}

(1) On doit décider si la phrase
<< $ \forall x \in \R, \exists y \in \R $
tel que $ x < xy $>>
est vrai.
La phrase est fausse:
il y a un \emph{contre-exemple}
à la proposition
<<$\exists y \in \R $
tel que $ x < xy $>>
.
En effet, considérons $ x=0 $.
Alors pour tout $ y \in \R $,
$ x = x y $ car $ 0 = 0y $,
donc il n'existe pas de $ y $
tel que $ x < xy $.

(2) On doit décider si la phrase
<< $ \forall x \in \R, 
x > 1$ ou $ x^2 < 2 $>>
est vrai.
Cette phrase est fausse.
Un contre-exemple est,
par exemple,
$ x = -2 $.
En effet, on a
$ x = -2 < -1 $
et $ x^2 = 4 > 2 $.

(3) On doit décider si la phrase
<< $ \forall x \in \R, 
x > 1 \Rightarrow x > 0 $>>
est vrai.
Cett phrase est vrai.
En effet, soit un $ x \in \R $.
Supposons $ x > 1 $.
Comme $ 1 > 0 $, on a que $ x > 0 $.

Les 2 exercices suivants
sont des applications
du raisonement par récurrence.
On a essayé de faire ressortir
la structure de la preuve,
qui est toujours la même:
énoncé de ce qu'on doit montrer,
initialisation, hérédité, conclusion.

\section{Exercice 5}

(1) On doit montrer que pour tout
$ n \in \N^{*} $,
\begin{equation}
	\sum_{k=1}^{n} k
	=
	\frac{n(n+1)}{2}
	.
\end{equation}
On va le démontrer par récurrence.
Appelons $ (H_{n}) $ la proposition:
\begin{equation}
	\sum_{k=1}^{n} k
	=
	\frac{n(n+1)}{2}
	.
\end{equation}

\textbf{Initialisation.}
On montre que $ (H_{n}) $
est vraie pour $ n=1 $.
D'une part,
\begin{equation}
	\sum_{k=1}^{1} k
	=
	1
	,
\end{equation}
et d'autre part,
\begin{equation}
	\frac{1(1+1)}{2}
	=
	\frac{2}{2}
	=
	1
	.
\end{equation}
La proposition $ (H_{1}) $ 
est donc vrai.

\textbf{Hérédité.}
Supposons que, pour un certain $ n \in \N^{*} $;
la proposition $ (H_{n}) $ soit vraie.
Démontrons $ (H_{n+1}) $.
On a:
\begin{align}
	\sum_{k=1}^{n+1}
	k
	=
	\sum_{k=1}^{n}
	k
	+
	(n+1)
	,
\end{align}
et par hypothèse de récurrence,
on a que
\begin{equation}
	\sum_{k=1}^{n} k
	=
	\frac{n(n+1)}{2}
	,
\end{equation}
donc en remplaçant on obtient:
\begin{align}
	\sum_{k=1}^{n+1}
	k
	=
	\frac{n(n+1)}{2}
	+
	n+1
	.
\end{align}
On simplifie la dernière expression:
\begin{align}
	\frac{n(n+1)}{2}
	+
	n+1
	&=
	\frac{n(n+1) + 2(n+1)}{2}
	\com{On met au même dénominateur}
	\\
	&=
	\frac{(n+1)(n+2)}{2}
	\com{On factorise par $ n+1 $  }
	.
\end{align}
On a donc prouvé:
\begin{equation}
	\sum_{k=1}^{n+1}
	k
	=
	\frac{(n+1)(n+2)}{2}
	,
\end{equation}
c'est à dire la proposition
$ (H_{n+1}) $.
On a bien prouvé l'hérédité,
c'est à dire que
$ (H_{n}) \Rightarrow (H_{n+1}) $.

\textbf{Conclusion.}
Par le principe de récurrence,
on a prouvé que
pour tout $ n \ge 1 $,
\begin{equation}
	\sum_{k=1}^{n} k
	=
	\frac{n(n+1)}{2}
	.
\end{equation}

(2) On doit montrer que pour tout
$ n \in \N^{*} $,
\begin{equation}
	\sum_{k=1}^{n} k^2
	=
	\frac{n(n+1)(2n+1)}{6}
	.
\end{equation}
On va le démontrer par récurrence.
Appelons $ (H_{n}) $ la proposition:
\begin{equation}
	\sum_{k=1}^{n} k^2
	=
	\frac{n(n+1)(2n+1)}{6}
	.
\end{equation}

\textbf{Initialisation.}
On montre que $ (H_{n}) $
est vraie pour $ n=1 $.
D'une part,
\begin{equation}
	\sum_{k=1}^{1} k^2
	=
	1
	,
\end{equation}
et d'autre part,
\begin{equation}
	\frac{1(1+1)(2 \times 1+3)}{6}
	=
	\frac{6}{6}
	=
	1
	.
\end{equation}
La proposition $ (H_{1}) $ 
est donc vrai.

\textbf{Hérédité.}
Supposons que, pour un certain $ n \in \N^{*} $;
la proposition $ (H_{n}) $ soit vraie.
Démontrons $ (H_{n+1}) $.
On a:
\begin{align}
	\sum_{k=1}^{n+1}
	k^2
	=
	\sum_{k=1}^{n}
	k^2
	+
	(n+1)^2
	,
\end{align}
et par hypothèse de récurrence,
on a que
\begin{equation}
	\sum_{k=1}^{n} k^2
	=
	\frac{n(n+1)(2n+1)}{6}
	,
\end{equation}
donc en remplaçant on obtient:
\begin{align}
	\sum_{k=1}^{n+1}
	k^2
	=
	\frac{n(n+1)(2n+1)}{6}
	+
	(n+1)^2
	.
\end{align}
On simplifie la dernière expression:
\begin{align}
	\frac{n(n+1)(2n+1)}{6}
	+
	(n+1)^2
	& =
	\frac{n(n+1)(2n+1 + 6(n+1)^2)}{6}
	\\
	& =
	\frac{(n+1)(n(2n+1) + 6(n+1))}{6}
	\\
	& =
	\frac{(n+1)(2n^2+n + 6n+6)}{6}
	\\
	& =
	\frac{(n+1)(2n^2+7n+6)}{6}
	,
\end{align}
ce qui n'est pas encore le résultat
que l'on souhaite; on voudrait
obtenir
\begin{equation}
	\frac{(n+1)(n+2)(2n+1)}{6}
	.
\end{equation}
Mais si on développe
$ (n+2)(2n+1) $
on trouve
\begin{equation}
	2n^2+7n+6
	,
\end{equation}
ce qui prouve bien que 
\begin{align}
	\frac{n(n+1)(2n+1)}{6}
	+
	(n+1)^2
	& =
	\frac{(n+1)(2n^2+7n+6)}{6}
	\\
	& =
	\frac{(n+1)(n+2)(2n+1)}{6}
	,
\end{align}
et donc
\begin{equation}
	\sum_{k=1}^{n+1}
	k^2
	=
	\frac{(n+1)(n+2)(2n+1)}{6}
	,
\end{equation}
c'est à dire que $ (H_{n+1}) $
est vraie.

On a bien prouvé l'hérédité,
c'est à dire que
$ (H_{n}) \Rightarrow (H_{n+1}) $.

\textbf{Conclusion.}
Par le principe de récurrence,
on a prouvé que
pour tout $ n \ge 1 $,
\begin{equation}
	\sum_{k=1}^{n} k^2
	=
	\frac{n(n+1)(2n+1)}{6}
	.
\end{equation}


(3) On doit montrer que pour tout
$ n \in \N^{*} $,
\begin{equation}
	\sum_{k=1}^{n} k^3
	=
	\left( 
	\frac{n(n+1)}{2}
	\right)^2
	.
\end{equation}
On va le démontrer par récurrence.
Appelons $ (H_{n}) $ la proposition:
\begin{equation}
	\sum_{k=1}^{n} k^3
	=
	\left( 
		\frac{n(n+1)}{2}
	\right)^2
	.
\end{equation}

\textbf{Initialisation.}
On montre que $ (H_{n}) $
est vraie pour $ n=1 $.
D'une part,
\begin{equation}
	\sum_{k=1}^{1} k^3
	=
	1
	,
\end{equation}
et d'autre part,
\begin{equation}
	\left(
		\frac{1(1+1)}{2}
	\right)
	^{2} 
	=
	\left(
		\frac{1}{1}
	\right)
	^2
	=
	1
	.
\end{equation}
La proposition $ (H_{1}) $ 
est donc vrai.

\textbf{Hérédité.}
Supposons que, pour un certain $ n \in \N^{*} $;
la proposition $ (H_{n}) $ soit vraie.
Démontrons $ (H_{n+1}) $.
On a:
\begin{align}
	\sum_{k=1}^{n+1}
	k^3
	=
	\sum_{k=1}^{n}
	k^3
	+
	(n+1)^3
	,
\end{align}
et par hypothèse de récurrence,
on a que
\begin{equation}
	\sum_{k=1}^{n} k^3
	=
	\left( 
		\frac{n(n+1)}{2}
	\right)^2
	,
\end{equation}
donc en remplaçant on obtient:
\begin{align}
	\sum_{k=1}^{n+1}
	k^3
	=
	\left( 
		\frac{n(n+1)}{2}
	\right)^2
	+
	(n+1)^3
	.
\end{align}
On simplifie la dernière expression:
\begin{align}
	\left( 
		\frac{n(n+1)}{2}
	\right)^2
	+
	(n+1)^3
	& =
	\frac{n^2(n+1)^2 + 4(n+1)^3}{4}
	\\
	& =
	\frac{(n+1)^2(n^2 + 4(n+1))}{4}
	\com{On met $ (n+1)^2 $  en facteur}
	\\
	& =
	\frac{(n+1)^2(n^2 + 4n+4)}{4}
	\\
	& =
	\frac{(n+1)^2(n+2)^2}{4}
	\com{On factorise l'identité remarquable}
	\\
	& = 
	\left( 
		\frac{(n+1)(n+2)}{2}
	\right)^2
	,
\end{align}
et on a donc prouvé
\begin{equation}
	\sum_{k=1}^{n+1}
	k^3
	=
	\left( 
		\frac{(n+1)(n+2)}{2}
	\right)^2
	,
\end{equation}
c'est à dire que $ (H_{n+1}) $
est vraie.

On a bien prouvé l'hérédité,
c'est à dire que
$ (H_{n}) \Rightarrow (H_{n+1}) $.

\textbf{Conclusion.}
Par le principe de récurrence,
on a prouvé que
pour tout $ n \ge 1 $,
\begin{equation}
	\sum_{k=1}^{n+1}
	k^3
	=
	\left( 
		\frac{(n+1)(n+2)}{2}
	\right)^2
	.
\end{equation}

\section{Exercice 6}

(1)
On doit montrer que pour tout
$ n \in \N $
on a
$ 2^{n} > n $.
On va le montrer par récurrence.
Soit $ (H_n) $ la proposition
\begin{equation}
2^{n} > n 
.
\end{equation}

\textbf{Initialisation.}
On montre que la propriété
est vrai pour $ n=0 $.
On a
$2^{0} = 1$
,
donc
\begin{equation}
	2^{0} > 0
	,
\end{equation}
ce qui veut dire que la propriété
$ (H_{0}) $
est vraie.
On prouve aussi $ (H_{1}) $,
car l'argument de l'hérédité
nécessitera $ n \ge 1 $.
On a $ 2^{1} = 2 $,
donc
\begin{equation}
	2^{1} > 1
	,
\end{equation}
et $ (H_{1}) $
est vraie.

\textbf{Hérédité.}
On suppose que pour un certain
$ n \ge 1 $,
la propriété
$ (H_{n}) $ est vraie,
c'est à dire que l'on a
\begin{equation}
	2^{n} > n
	.
\end{equation}
On va montrer que
$ (H_{n+1}) $
est aussi vraie.
On a
\begin{equation}
	2^{n+1}
	=
	2 \times
	2^{n}
	,
\end{equation}
et par hypothèse
de récurrence
$ 2^{n} > n $,
donc
\begin{equation}
	2^{n+1}
	=
	2 \times
	2^{n} 
	>
	2 n
	.
\end{equation}
De plus on a
\begin{equation}
	2n \ge n+1
	,
\end{equation}
car
\begin{equation}
	2n \ge n+1
	\eq
	n \ge 1,
\end{equation}
et donc
\begin{equation}
	2^{n+1}
	> 2n
	\ge n+1
	,
\end{equation}
donc $ 2^{n+1} > n+1 $
et la propriété $ (H_{n+1}) $
est démontrée.
On a bien prouvé l'hérédité.

\textbf{Conclusion.}
Par le principe de récurrence,
on a prouvé que pour tout
$ n \in \N $,
\begin{equation}
	2^{n} 
	>
	n
	.
\end{equation}

(2) On doit prouver l'inégalité
de Bernouilli; pour tout $ x \ge 0 $,
pour tout $ n \in \N $ on a
\begin{equation}
	(1+x)^{n} 
	\ge
	1 + nx
	.
\end{equation}
On fixe un $ x \ge 0 $
et on le prouve par récurrence
sur $ n \in \N $.
Soit $ (H_{n}) $ la proposition
\begin{equation}
	(1+x)^{n} 
	\ge
	1 + nx
	.
\end{equation}

\textbf{Initialisation.}
On prouve $ (H_{0}) $.
On a $ (1+x)^{0} = 1 $
et $ 1 + 0x = 1 $
donc $ (H_{0}) $ est vraie.

\textbf{Hérédité.}
On suppose que pour un certain
$ n \ge 1 $,
la propriété
$ (H_{n}) $ est vraie,
c'est à dire que l'on a
\begin{equation}
	(1+x)^{n} 
	\ge
	1 + nx
	.
\end{equation}
On montre $ (H_{n+1}) $.
On a
\begin{equation}
	(1+x)^{n+1}
	=
	(1+x)^{n} 
	(1+x)
	,
\end{equation}
et par hypothèse de récurrence
$ (1+x)^n \ge 1 + nx $,
donc
\begin{equation}
	(1+x)^{n+1}
	=
	(1+x)^{n} 
	(1+x)
	\ge
	(1+nx)(1+x)
	=
	1 + x + nx + nx^2
	=
	1 + (n+1)x + nx^2
	,
\end{equation}
et comme $ nx^2 $
est positif, on a
\begin{equation}
	1 + (n+1)x + nx^2
	\ge
	1 + (n+1)x
	,
\end{equation}
donc finalement,
\begin{equation}
	(1+x)^{n+1}
	\ge
	1 + (n+1)x
	.
\end{equation}
On a bien prouvé $ (H_{n+1}) $.

\textbf{Conclusion.}
Par le principe de récurrence
on a prouvé que pour tout $ n \in \N $,
\begin{equation}
	(1+x)^{n}
	\ge
	1 + nx
	.
\end{equation}

\section{Exercice 7}

Pour tout cet exercice,
on se fixe
deux ensembles
$ X $ et $ Y $,
une application
$ f : X \to Y $,
des sous-ensembles
$ A,B \subset X $
et $ C,D \subset Y $.

(1) On doit montrer:
\begin{equation}
	A \subset B
	\Rightarrow
	f(A)
	\subset
	f(B)
	,
\end{equation}
c'est à dire que
l'on suppose que $ A \subset B $
et on doit en déduire
$ f(A) \subset f(B) $.

On rappelle que l'énoncé
<<
$ A \subset B $
>>
signifie:
<<
si $ x \in A $ alors $ x \in B $.
>>
De même, l'énoncé
<<
$ f(A) \subset f(B) $
>>
signifie:
<<
si $ y \in f(A) $
alors $ y \in f(B) $.
>>

On suppose donc que
$ A \subset B $.
On veut montrer
<<
si $ y \in f(A) $
alors $ y \in f(B) $.
On fixe donc un
$ y \in f(A) $.
On se souvient que
$ f(A) $ est l'ensemble
\begin{equation}
	\left\{ 
		f(x);
		x \in A
	\right\}
	.
\end{equation}
C'est un sous-ensemble
de $ Y $,
c'est à dire l'ensemble d'arrivé
de $ f $.
Si $ y \in f(A) $,
cela veut dire qu'il existe
un $ x \in A $ tel que
$ y = f(x) $.
Comme $ x \in A $
et que $ A \subset B $,
on a que $ x \in B $.
Donc $ y = f(x) \in f(B) $,
car on se rappelle que
$ f(B) $ est l'ensemble
\begin{equation}
	\left\{ 
		f(x);
		x \in B
	\right\}
	.
\end{equation}

(2) On doit montrer:
\begin{equation}
	f(A \cup B)
	=
	f(A)
	\cup
	f(B)
	.
\end{equation}
Pour montrer l'égalité
de 2 ensembles,
la méthode est toujours la même:
on montre la double inclusion,
c'est à dire dans notre cas
\begin{align}
	f(A \cup B)
	&  \subset
	f(A)
	\cup
	f(B)
	\\
	& \text{et}
	\\
	f(A)
	\cup
	f(B)
	& \subset
	f(A \cup B)
	.
\end{align}

Montrons la première inclusion,
\begin{equation}
	f(A \cup B)
	 \subset
	f(A)
	\cup
	f(B)
	.
\end{equation}
Soit $ y \in f( A \cup B) $.
Cela veut dire qu'il existe
$ x \in A \cup B $
tel que
$ y = f(x) $.
Le fait que $ x \in A \cup B $
veut dire
<<
$x \in A$
ou
$x \in B$
>>.
Il faut considérer les 2 cas.
\begin{itemize}
	\item
		Dans le premier cas
		$ x \in A $.
		Donc
		$ y = f(x) \in f(A) $,
		et alors
		$ y \in f(A) \cup f(B) $.
	\item
		Dans le deuxième cas,
		$ x \in B $.
		Donc
		$ y = f(x) \in f(B) $,
		et alors
		$ y \in f(A) \cup f(B) $.
\end{itemize}
(On se rappelle que
en général, pour deux ensemble
$ C $ et $ D $ on a
$ C \subset C \cup D$ 
et
$ D \subset C \cup D$.)
Dans le 2 cas, on a la conclusion voulu,
c'est à dire que
$ y \in f(A) \cup f(B) $.
On peut conclure que
$ f(A \cup B) \subset f(A) \cup f(B) $.

Montrons la deuxième inclusion,
\begin{equation}
	f(A)
	\cup
	f(B)
	\subset
	f(A \cup B)
	.
\end{equation}
On va utiliser la propriété suivante:
pour trois ensembles $ C, D, E $
on a
\begin{align}
	C & \subset E
	\\
	& \text{et}
	\\
	D & \subset E
	\\
	& \Rightarrow
	\\
	C \cup D & \subset E
	,
\end{align}
(le démontrer si ce n'est pas clair pour vous).

On montre donc d'abord $ f(A) \subset f(A \cup B) $.
On a $ A \subset A \cup B $,
et d'après la question (1)
cela implique
$ f(A) \subset f(A \cup B)$.

De même,
on a $ B \subset A \cup B $,
et d'après la question (1)
cela implique
$ f(B) \subset f(A \cup B)$.

D'après la propriété énoncé plus haut,
cela implique
\begin{equation}
	f(A)
	\cup f(B)
	\subset
	f( A \cup B)
	.
\end{equation}

\textbf{Conclusion.}
On a montré 
$ f(A \cup B) \subset f(A) \cup f(B) $
et
$
f(A)
\cup f(B)
\subset
f( A \cup B)
$,
cela veut dire que
$ f(A \cup B) = f(A) \cup f(B) $
.

(3)
On doit montrer
\begin{equation}
	f(A \cap B)
	\subset
	f(A)
	\cap
	f(B)
	.
\end{equation}
Soit $ y \in f( A \cap B) $.
Cela veut dire qu'il existe
$ x \in A \cap B $
tel que
$ y = f(x) $.
Comme $ x \in A \cap B $,
$ x \in A $,
donc $ y = f(x) \in f(A) $.
Comme $ x \in A \cap B $,
$ x \in B $,
donc $ y = f(x) \in f(B) $.
Comme $ y \in f(A) $
et $ y \in f(B) $
on en déduit que
$ y \in f(A) \cap f(B) $.

(4) On rappelle que
\begin{equation}
	f^{-1}(C)
	=
	\left\{ 
	x \in X
	\tq
	f(x) \in C
	\right\}
	.
\end{equation}
C'est un sous-ensemble de
$ X $ (c'est à dire l'ensemble de départ de $ f $).
On a la caractérisation:
\begin{align}
	x \in	
	f^{-1}(C)
	\eq
	f(x) \in C
	.
\end{align}

On doit montrer:
\begin{align}
	C \subset D
	% \\
	\Rightarrow
	% \\
	f^{-1} (C)
	\subset
	f^{-1} (D)
	.
\end{align}
On suppose donc
$ C \subset D $
et
on se donne un 
$ x \in f^{-1} (C) $.
On a alors
$ f(x) \in C $.
Comme $ C \subset D $,
on a aussi
$ f(x) \in D $.
Cela veut dire que
$ x \in f^{-1} (D) $.
On a bien montré
$ f^{-1} (C) \subset f^{-1} (D) $.

(5) On doit montrer:
\begin{equation}
	f^{-1} (C \cup D)
	=
	f^{-1} (C) \cup f^{-1}(D)
	.
\end{equation}
On montre la double-inclusion.
On commence par montrer que
$ 
f^{-1} (C \cup D)
\subset
f^{-1} (C) \cup f^{-1}(D)
$:
soit
$ x \in
f^{-1} (C \cup D)$.
On a
$ f(x) \in C \cup D $,
c'est à dire que
$ f(x) \in C$
ou
$ f(x) \in D $
On traite les deux cas:
\begin{itemize}
	\item Si $ f(x) \in C $
		alors $ x \in f^{-1}(C) $
		et donc
		$ x \in f^{-1} (C) \cup f^{-1} (D) $.
	\item Si $ f(x) \in D $
		alors $ x \in f^{-1}(D) $
		et donc
		$ x \in f^{-1} (C) \cup f^{-1} (D) $.
\end{itemize}
Dans les deux cas,
$ x \in f^{-1} (C) \cup f^{-1} (D) $.
On a prouvé que
$
f^{-1} (C \cup D)
\subset
f^{-1} (C) \cup f^{-1}(D)
$.

On montre la deuxième inclusion,
c'est à dire
$ 
f^{-1} (C) \cup f^{-1}(D)
\subset
f^{-1} (C \cup D)
$.
Soit $ x \in
f^{-1} (C) \cup f^{-1}(D)
$:
\begin{itemize}
	\item 
		Si $ x \in f^{-1} (C) $
		alors $ f(x) \in C $
		donc $ f(x) \in C \cup D $
		et finalement
		$ x \in f^{-1} (C \cup D) $.
	\item 
		Si $ x \in f^{-1} (D) $
		alors $ f(x) \in D $
		donc $ f(x) \in C \cup D $
		et finalement
		$ x \in f^{-1} (C \cup D) $.
\end{itemize}
Dans les deux cas,
on a 
$ x \in f^{-1} (C \cup D) $.
On a bien montré que
$ 
f^{-1} (C) \cup f^{-1}(D)
\subset
f^{-1} (C \cup D)
$.

(6) On doit montrer que
\begin{equation}
	f^{-1} ( C \cap D )
	=
	f^{-1} (C)
	\cap
	f^{-1} (D)
	.
\end{equation}
Pour montrer cet égalité entre ensembles,
on va montrer que

\begin{equation}
	x \in
	f^{-1} ( C \cap D )
	\eq
	x \in
	f^{-1} (C)
	\cap
	f^{-1} (D)
	.
\end{equation}
On a:
\begin{align}
	x
	& \in
	f^{-1} ( C \cap D )
	\\
	& \eq
	\\
	f(x)
	& \in
	C \cap D
	\\
	& \eq
	\\
	f(x)
	\in
	C
	& \text{ et }
	f(x)
	\in
	D
	\\
	& \eq
	\\
	x
	\in
	f^{-1} (C)
	& \text{ et }
	x
	\in
	f^{-1} (D)
	\\
	& \eq
	\\
	x
	& \in
	f^{-1} (C)
	\cap
	f^{-1} (D)
	.
\end{align}
Cela prouve
$ 
f^{-1} ( C \cap D )
=
f^{-1} (C)
\cap
f^{-1} (D)
$.

\section{Exercice 9}

On rappelle que pour
déterminer si un repère
$ (A, u, v, w) $
est direct
ou indirect,
il faut calculer
le produit-mixte
$ det(u, v, w) $;
si celui-ci est strictement
positif le repère est direct,
si celui-ci est strictement
negatif le repère est indirect.

On rappelle que le produit
mixte $ det(u,v,w) $
est défini par:
\begin{equation}
	det(u,v,w)
	=
	(u \wedge v) \cdot w
	,
\end{equation}
où $ u \wedge v $ est le produit vectoriel
de $ u $ et $ v $
et <<$\cdot$>> est le produit scalaire.
Résumons les propriétés essentielles
du produit vectoriel:
\begin{itemize}
	\item $ u \wedge v = - v \wedge u $
		(antisymétrie)
	\item $ u \wedge u = 0 $ 
	\item $ (u + v) \wedge w = u \wedge w + v \wedge w $ 
		et
		$ u \wedge (v + w) = u \wedge w + u \wedge w $ 
		(linéarité)
	\item $ u \wedge v $ est orthogonal (perpendiculaire) à
		$ u $ et à $ v $
	\item si $ (i,j,k) $ est la base canonique on a:
		\begin{itemize}
			\item $ i \wedge j = k $
			\item $ j \wedge k = i $
			\item $ k \wedge i = j $ 
		\end{itemize}
\end{itemize}
On se souviendra de la dernière propriété
avec le dessin suivant

\begin{tikzcd}
	i \arrow[rr, bend left] &                         & j \arrow[ld, bend left] \\
			 & k \arrow[lu, bend left] &                        
\end{tikzcd}	
qu'il faut lire ainsi:
je parcours le cercle dans le sens indiqué
par les flêches,
le produit vectoriel des deux premiers
vecteurs que je rencontre est donné
par le troisième vecteur.
Si je parcours le cercle dans l'autre sens,
c'est la même chose mais je met un signe moins
au résultat.

Déterminons si le repère
$ (0, i, j, k) $
est direct ou indirect.
On rappelle que $ (i,j,k) $
est la base canonique.
On utilise la caractérisation avec le produit mixte.
On calcule:
\begin{align}
	det(i, j, k)
	& =
	(i \wedge j) \cdot k
	\\
	&=
	k \cdot k \com{car $ i \wedge j = k $}
	\\
	& =
	1 \com{car $k \cdot k = \norm{k}^2 = 1$}
	,
\end{align}
donc le produit mixte
est strictement positif,
donc le repère est direct.

Déterminons si le repère
$ (A_{(1,1,1)} , i, j, k) $
est direct ou indirect.
Ce repère a la même base
que le repère précédent
(seul l'origine change);
or la propriété d'être direct
ou indirect ne dépend que de la
base.
Le repère est donc direct.

Déterminons si le repère
$ (0, i, k, j) $
est direct ou indirect.
On utilise la caractérisation avec le produit mixte.
On calcule:
\begin{align}
	det(i, k, j)
	& =
	(i \wedge k) \cdot j
	\\
	&=
	-j \cdot j \com{car $ i \wedge k = -j $}
	\\
	& =
	-1 \com{car $j \cdot j = \norm{j}^2 = 1$}
	,
\end{align}
donc le produit mixte
est strictement négatif,
donc le repère est indirect.

Déterminons si le repère
$ (0, i+j, j+k, k) $
est direct ou indirect.
On utilise la caractérisation avec le produit mixte.
On calcule:
\begin{align}
	det(i+j, j+k, k)
	& =
	((i+j) \wedge (j+k)) \cdot k
	\\
	& =
	(i \wedge (j+k)
	+
	j \wedge (j+k)
	)
	\cdot k
	\com{par linéarité}
	\\
	& =
	(i \wedge j
	+
	i \wedge k
	+
	j \wedge j
	+
	j \wedge k
	)
	\cdot k
	\com{par linéarité}
	\\
	& =
	(i \wedge j
	+
	i \wedge k
	+
	j \wedge k
	)
	\cdot k
	\com{car $ j \wedge j = 0 $ }
	\\
	& =
	(k
	-j
	+
	i
	)
	\cdot k
	\com{car $ i \wedge j = k $,
		$ i \wedge k = -j $
		et
		$ j \wedge k =i $ 
	}
	\\
	& =
	k \cdot k
	-j \cdot k
	+
	i \cdot k
	\com{par linéarité}
	\\
	& =
	k \cdot k
	\com{car $ j \perp k $ et $ i \perp k $  }
	\\
	& = 1
	\com{car $ k \cdot k = \norm{k}^2=1 $}
	,
\end{align}
donc le produit mixte
est strictement positif,
donc le repère est direct.
On a utilisé le fait que
si deux vecteurs $ u $
et $ v $ sont orthogonaux
(ce que l'on note $ u \perp v $)
alors leur produit scalaire est nul,
c'est à dire $ u \cdot v = 0 $.

\bibliographystyle{alpha}
\bibliography{../references}
\end{document}
