%st&pdflatex
\documentclass[12pt, a4paper,oneside]{article} %article pour voir la todolist
% \documentclass[12pt, a4paper,oneside]{amsart} %amsart pour faire joli
\usepackage{amssymb, amsfonts, mathrsfs}
%\usepackage{a4wide, enumerate,} %test
% \usepackage{showkeys} % for work version
\usepackage[textwidth=1in,backgroundcolor=white,linecolor=white,bordercolor=white,textcolor=red]{todonotes}
%\usepackage{a4wide}

\usepackage{glossaries}
\usepackage[bookmarks]{hyperref} % in that order, to not have a link with every
				% abreviation
\usepackage{tikz}
\usepackage{tikz-cd}	

\newcommand{\g}[1]{\gls{#1}}

\newglossaryentry{graphs} {
	name={\ensuremath{\Theta}},
	description={the compact family of graphs}
}
\newglossaryentry{graphs_dual} {
	name={\ensuremath{\Theta^{*} }},
	description={the compact family of dual graphs}
}

\input{../these/preamble.tex}

\title{Corrections feuille 1 L2 SV}
\author{Florestan {Martin-Baillon}}

\begin{document}

\maketitle

\section{Exercice 4}

Pour chacune des suites suivantes, dire si elle
est croissante, décroissante, minorée, majorée,
bornée.

\textbf{a)}
La suite
$ u_{n} = \frac{n}{n+1}, n \in \N $.
\begin{itemize}
	\item
		Comme la suite est positive,
		pour savoir si elle est croissante
		on calcule $ \frac{u_{n+1} }{u_{n} } $
		et on le compare à $ 1 $.
		On a pour $ n \ge 1 $:
		\begin{align}
		\frac{u_{n+1} }{u_{n} }
		=
		\frac{n+1}{n+2}
		\cdot
		\frac{n+1}{n}
		=
		\frac{(n+1)^{2} }{n(n+2)}
		=
		\frac{n^2 + 2n + 1}{n^2 + 2n}
		,
		\end{align}
		donc le numérateur est strictement
		plus grand
		que le dénominateur
		et la fraction est toujours
		strictement plus grande
		que $ 1 $.
		Cela veut dire que la suite
		est strictement croissante.
	\item Comme la suite est strictement croissante,
		elle n'est pas décroissante.
	\item La suite $ (u_{n}) $ est positive,
		car c'est un quotient de deux
		quantités positives,
		donc elle est minorée
		(par $ 0 $).
	\item Comme on a $ n \le n+1 $
		pour tout $ n \in \N $,
		on a $ u_{n} \le 1 $.
		Donc la suite est majorée.
	\item La suite est majorée et minorée,
		elle est donc bornée.
\end{itemize}
On aurait pu aussi raisonner autrement.
La suite $ (u_{n}) $ tend vers $ 1 $,
en effet:
\begin{equation}
	u_{n}
	=
	\frac{n}{n+1}
	=
	\frac{1}{1 + \frac{1}{n}}
	\to_{n \to \infty} 
	\frac{1}{1}
	= 1.
\end{equation}
La suite $ (u_{n}) $
est donc convergente,
donc bornée
(donc majorée et minorée).

\textbf{b)}
La suite
$ v_{n} = (-2)^{n}, n \in \N $ 
.
On remarque que pour les $ n $
pair, $ v_{n} $ est positif;
en effet, pour tout $ n \in \N  $ :
\begin{equation}
	v_{2n} =
	(-2)^{2n}
	=
	(-1)^{2n}
	2^{2n} 
	=
	((-1)^{2})^{n} 
	2^{2n} 
	=
	1^{n} 
	\times
	2^{2n} 
	=
	2^{2n} 
	\ge 0
	,
\end{equation}
tandis que pour les
$ n $ impair,
$ v_{n} $ est négatif;
en effet, pour tout $ n \in \N $ :
\begin{equation}
	v_{2n+1} =
	(-2)^{2n+1}
	=
	(-1)^{2n+1}
	2^{2n=1} 
	=
	-(-1)^{2n} 
	2^{2n} 
	=
	- 
	2^{2n} 
	\le 0
	,
	.
\end{equation}
On en déduit que la suite n'est ni
croissante, ni décroissante:
\begin{itemize}
	\item si elle était croissante,
		comme $ v_{0} = 1 $
		est positif,
		les termes de la suite
		seraient toujours positifs,
		ce qui n'est pas le cas
		car on a vu que les termes
		d'indices impairs étaient négatifs.
	\item si elle était décroissante,
		comme $ v_{1} = -2 $
		est négatif,
		les termes de la suite
		seraient toujours négatifs,
		ce qui n'est pas le cas
		car on a vu que les termes
		d'indices pairs étaient positifs.
\end{itemize}

La suite n'est pas minorée.
En effet, considérons la suite 
extraite
$ v_{2n+1} = -2^{2n+1}, n \in \N $.
Cette suite tend vers $ - \infty $,
donc n'est pas minorée,
donc $ (v_{n}) $ aussi n'est pas minorée.

La suite n'est pas majorée.
En effet, considérons la suite 
extraite
$ v_{2n} = 2^{2n}, n \in \N $.
Cette suite tend vers $ + \infty $,
donc n'est pas majorée,
donc $ (v_{n}) $ aussi n'est pas majorée.

\textbf{c)}
La suite
$ w_{n} = \frac{(-1)^{n} }{n}, n \in \N^{*} $.
On remarque, comme pour la
question \textbf{b)}
que les termes d'indice pair
sont positifs
tandis que les termes d'indice 
impair sont négatif
(avec la même démonstration).
On en déduit de la même manière
que la suite n'est ni
croissante, ni décroissante.

La suite converge vers $ 0 $: en effet on a
\begin{equation}
	\abs{w_{n}} = \frac{1}{n}
	\to_{n \to \infty} 0,
\end{equation}
et comme $ (\abs{w_{n}}) $
tend vers $ 0 $,
$ (w_{n})  $ tend vers $ 0 $.
Comme la suite est convergente,
elle est bornée et donc majorée
et minorée.

\textbf{d)}
La suite
$ x_{n} = \frac{n^{2} + 1}{n + 1}, n \in \N $.
Pour étudier la monotonie de cette suite,
on va étudier la fonction annexe
\begin{equation}
	f(x) = \frac{x^{2} +1}{x+1}
	,
	x \in \R_{+}.
\end{equation}
Comme on a $ x_{n} = f(n) $,
si la fonction $ f $
est monotone,
la suite $ (x_{n}) $
le sera aussi.
On calcule la dérivée de $ f $,
pour $ x \in \R_{+} $:
\begin{equation}
	f'(x)=
	\frac{2x(x+1) - (x^2+1)}{(x+1)^{2} }
	=
	\frac{2x^2+2x-x^2-1}{(x+1)^2}
	=
	\frac{x^2+2x-1}{(x+1)^{2}}
	.
\end{equation}
Le numérateur est un polynome du second
degré.
Son discriminant
est
$ \Delta = 4 + 4 = 8 > 0 $,
donc ses racines sont
\begin{equation}
	x_{1} 
	= -1-\sqrt{2}
		,
	x_{2}  
	= -1+\sqrt{2}
	.
\end{equation}
On en déduit que ce polynome
est positif pour
$ x \ge x_{2} $,
donc que $ f' $
est positive
pour $ x \ge x_{2} $.
On a
\begin{equation}
	1 \le \sqrt{2} \le 2
	,
\end{equation}
donc
\begin{equation}
	0 \le -1 + \sqrt{2} \le 1
	,
\end{equation}
donc $ f' $ est strictement
positive
pour $ x \ge 1 $.
On en déduit que $ f $
est strictement croissante
pour $ x \ge 1 $.
Donc $ (x_{n}) $
est une suite strictement croissante
pour $ n \ge 1 $.
Comme on a $ x_{0} = 1 $
et $ x_{1} =1 $,
on a que $ (x_{n}) $
est croissante pour
$ n \ge 0 $.
Comme $ (x_{n}) $
est strictement croissante
pour $ n \ge 1 $,
elle n'est pas décroissante.

La suite $ (x_n) $
est positive, donc minorée.
Elle tend vers $ +\infty $,
en effet:
\begin{equation}
	x_{n}=
	\frac{n^{2} +1}{n+1}
	=
	\frac{n + \frac{1}{n} }{1 + \frac{1}{n}}
	,
\end{equation}
et le numérateur tend vers $ +\infty $
tandis que le dénominateur tend vers $ 1 $,
donc la fraction tend vers $ +\infty $.
La suite n'est donc pas bornée.

\textbf{e)}
La suite
$ y_{n} = n e^{-n}, n \in \N $.
On a $ y_{0} = 0 $
et $ y_{n} > 0 $
pour
$ n \ge 1 $.
Pour étudier la monotonie de
la suite, on étudie le
quotient $ \frac{y_{n+1} }{y_{n} } $
pour $ n \ge 1 $.
On a:
\begin{equation}
	\frac{y_{n+1} }{y_{n} }
	=
	\frac{n+1}{n}
	\cdot
	\frac{e^{-(n+1)} }{e^{-n} }
	=
	\frac{n+1}{n}
	e^{-1}
	,
\end{equation}
et d'après la question
\textbf{a)}
on sait que la suite
$ \frac{n+1}{n} $
est décroissante,
et pour $ n=1 $
elle vaut
$ 2 $.
Donc on a que
$ \frac{n+1}{n} \le 2 $
pour tout $ n \ge 1 $.
Comme $ e > 2 $,
il vient que $ \frac{y_{n+1} }{y_{n} } < 1 $
pour tout $ n \ge 1 $.
La suite $ (y_n) $ est donc
décroissante pour $ n \ge 1 $.
Par contre comme
$ y_{0} = 0 < y_{1} $,
la suite n'est pas décroissante
pour $ n \ge 0 $.
Elle n'est pas non plus croissante.

Comme la suite est positive, elle est minorée.
Comme elle est décroissante à partir de $ n=1 $,
on a $ y_{n} \le y_{1} $ pour tout $ n \ge 1 $,
donc la suite est majorée.
Comme elle est majorée et minorée, elle est bornée.

\section{Exercice 5}

La suite $ (u_{n})  $ 
est définie par
$ u_{0} = 1 $ 
et
$ u_{n+1} = u_{n} + \frac{1}{(n+1)!} $ 
.

\textbf{a)}
Pour montrer que la suite
est croissante,
on calcule
$ u_{n+1} - u_{n} $.
Comme on a,
pour tout $ n \ge 1 $
\begin{equation}
	u_{n+1} - u_{n}
	=
	\frac{1}{(n+1)!}
	> 0
	,
\end{equation}
la suite est (strictement) croissante.

\textbf{b)}
Définissons,
pour $ n \ge 1 $
la propriété
$ (H_{n}): \frac{1}{n!} \le \frac{1}{2^{n-1} } $.

\textbf{Initialisation.}
pour $ n=1 $ on a
$ n! = 1 $ et
$ \frac{1}{2^{n-1} } = \frac{1}{2^{0} } = 1 $.
La propriété $ (H_{1}) $ est bien vérifiée.

\textbf{Hérédité.}
Supposons que pour un certain
$ n \ge 1 $, 
la propriété $ (H_{n}) $
soit vraie.
Démontrons la propriété
$ (H_{n+1)}) $.
On a
$ (n+1)! = n!  (n+1) $
et par hypothèse de
récurrence,
$ \frac{1}{n!} \le \frac{1}{2^{n-1} } $.
Donc:
\begin{equation}
	\frac{1}{(n+1)!}
	=
	\frac{1}{n! (n+1)}
	\le
	\frac{1}{2^{n-1} (n+1)}
	,
\end{equation}
et comme pour $ n \ge 1 $,
$ n+1 \ge 2 $, on a
\begin{equation}
	\frac{1}{2^{n-1} (n+1)}
	\le
	\frac{1}{2^{n-1} \times 2}
	=
	\frac{1}{2^{n} }
	.
\end{equation}
On a donc: $ \frac{1}{(n+1)!} \le \frac{1}{2^{n} } $
et la propriété $ (H_{n+1}) $ est vraie.

\textbf{Conclusion.}
Par le principe de récurrence,
pour tout $ n \ge 1 $ on a
\begin{equation}
	\frac{1}{n!}
	\le
	\frac{1}{2^{n-1} }
	.
\end{equation}

\textbf{c)}
D'après le résultat
de la question \textbf{b)}
on a, pour tout $ n \ge 1 $:
\begin{equation}
	u_{n+1}
	=
	u_{n} 
	+
	\frac{1}{(n+1)!}
	\le
	u_{n}
	+
	\frac{1}{2^{n} }
	.
\end{equation}
On va déduire
de cela,
par récurrence,
que pour tout $ n \ge 1 $:
\begin{equation}
	u_{n}
	\le
	u_{0}
	+
	\sum_{k=0}^{n-1} 
	\frac{1}{2^{k} }
	.
\end{equation}
En effet, posons
pour $ n \ge 1 $,
\begin{equation}
	(H_{n}):
	u_{n}
	\le
	u_{0}
	+
	\sum_{k=0}^{n-1} 
	\frac{1}{2^{k} }
	.
\end{equation}

\textbf{Initialisation.}
On a d'une part
$ u_{1} = 1 $ 
et d'autre part
$ u_{0} + \sum_{k=0}^{0} \frac{1}{2^{k} } = 1 + 1 = 2 $,
donc $ (H_{1}) $ est vraie.

\textbf{Hérédité.}
Supposons que pour un certain
$ n \ge 1 $, 
la propriété $ (H_{n}) $
soit vraie.
Démontrons la propriété
$ (H_{n+1)}) $.
On a
\begin{equation}
	u_{n+1}
	\le
	u_{n}
	+
	\frac{1}{2^{n} }
	,
\end{equation}
et par hypothèse de récurrence,
\begin{equation}
	u_{n}
	\le
	u_{0}
	+
	\sum_{k=0}^{n-1} 
	\frac{1}{2^{k} }
	,
\end{equation}
donc
\begin{equation}
	u_{n+1}
	\le
	u_{0}
	+
	\sum_{k=0}^{n-1} 
	\frac{1}{2^{k} }
	+
	\frac{1}{2^{n} }
	=
	\sum_{k=0}^{n} 
	\frac{1}{2^{k} }
	,
\end{equation}
ce qui veut dire 
que la propriété
$ (H_{n+1}) $ 
est vraie.

\textbf{Conclusion.}
On a montré par récurrence que
pour tout $ n \ge 1 $,
$
u_{n}
\le
u_{0}
+
\sum_{k=0}^{n-1} 
\frac{1}{2^{k} }
$.

Calculons la somme qui apparait
au membre de droite:
\begin{equation}
	\sum_{k=0}^{n-1} 
	\frac{1}{2^{k} }
	=
	\frac{1 - \frac{1}{2^{n} }}
	{1 - \frac{1}{2}}
	=
	\frac{1 - \frac{1}{2^{n}} }
	{\frac{1}{2}}
	=
	2
	(1 - \frac{1}{2^{n}})
	\le
	2
	,
\end{equation}
car
$
(1 - \frac{1}{2^{n}}) \le 1
$.
On en déduit que pour
tout $ n \ge 1 $,
\begin{equation}
	u_{n}
	\le
	u_{0}
	+
	\sum_{k=0}^{n-1} 
	\frac{1}{2^{k} }
	\le
	u_{0}
	+ 2
	= 1 + 2 = 3
	.
\end{equation}

\textbf{e)}
La suite $ (u_{n}) $ 
est croissante
et majorée
donc converge.




\bibliographystyle{alpha}
\bibliography{../references}
\end{document}
